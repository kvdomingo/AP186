\documentclass[12pt,a4paper]{article}
\usepackage{physics}
\usepackage{amssymb}
\usepackage{subcaption}
\usepackage{colortbl}
\usepackage{musicography}
\newcommand{\activity}{Activity 14 -- Logistic regression}
\input{spp.dat}

\begin{document}

\title{\TitleFont \activity}
\author[ ]{\textbf{Kenneth V. Domingo} \\
2015--03116 \\
App Physics 186, 1\textsuperscript{st} Semester, A.Y. 2019--20}
\affil[ ]{\corremail{kvdomingo@up.edu.ph} }

\maketitle
\thispagestyle{titlestyle}

\section*{Results and Discussion}
\setcounter{section}{1}

For this activity \cite{soriano}, I used the bananadat dataset from \cite{fruits} which has 273 images of bananas separated by underripe, midripe, yellowish-green, and overripe labels.

\subsection{Feature extraction: RGB}
For the training set I decided to take only images of underripe (green) and overripe (brownish-yellow) bananas and assign them class numbers 0 and 1, respectively. The feature vector extracted from the images consist only of the mean of each color channel, without any further preprocessing or segmentation. I simply reused my code from the previous activity, with minor modifications


\clearpage
\begin{table}[!htb]
	\centering
	\caption{Self-evaluation.}
	\begin{tabular}{||r|c||}
		\hline
		Technical correctness & 5 \\ \hline
		Quality of presentation & 5 \\ \hline
		Initiative & 0 \\ \hline
		\textbf{TOTAL} & \textbf{10} \\ \hline
	\end{tabular}
	\label{tab:self-eval}
\end{table}

\bibliographystyle{spp-bst}
\bibliography{biblio}

\end{document}