\documentclass[12pt,a4paper]{article}
\usepackage{physics}
\usepackage{amssymb}
\usepackage{subcaption}
\usepackage{colortbl}
\usepackage{musicography}
\newcommand{\activity}{Activity 13 -- Perceptron}
\input{spp.dat}

\begin{document}

\title{\TitleFont \activity}
\author[ ]{\textbf{Kenneth V. Domingo} \\
2015--03116 \\
App Physics 186, 1\textsuperscript{st} Semester, A.Y. 2019--20}
\affil[ ]{\corremail{kvdomingo@up.edu.ph} }

\maketitle
\thispagestyle{titlestyle}

\section*{Results and Discussion}
\setcounter{section}{1}

For this activity \cite{soriano}, I used the features extracted from the fruits (50 each of apples, mangoes, bananas) from the previous activity. The feature space in $a^*$-$b^*$ (obtained from the $L^*a^*b^*$ color space) is shown in Fig. \ref{fig:ab-space}. Since we'll be working with a linear classifier for now, we need to process only two classes at a time. We design the perceptron so that it follows a simple weight update rule

\begin{equation}\label{eq:weight-update}
	\Delta w_j = \eta \qty(y^i - z^i)x_j^i
\end{equation}

\noindent where $y$ is the ground truth label, $z$ is the predicted label, and $\eta$ is the learning rate, which we set to $10^{-2}$. The perceptron is trained for 100 epochs or until the sum of squares error (SSE) drops below some selected tolerance $\epsilon = 10^{-6}$ and the decision boundary is obtained from the final weights. The decision boundary and decision contours for each class pair is shown in Fig. \ref{fig:boundaries}.

\begin{figure}[htb]
	\centering
	\includegraphics[width=0.6\textwidth]{ab-space.png}
	\caption{Feature space in $a^*$-$b^*$.}
	\label{fig:ab-space}
\end{figure}

\begin{figure}[htb]
	\centering
	\begin{subfigure}[h!]{0.3\textwidth}
		\centering
		\includegraphics[width=\textwidth]{ban-app_decision.png}
		\caption{banana-apple}
		\label{fig:banana-apple}
	\end{subfigure}
	\begin{subfigure}[h!]{0.3\textwidth}
		\centering
		\includegraphics[width=\textwidth]{ban-ora_decision.png}
		\caption{banana-orange}
		\label{fig:banana-orange}
	\end{subfigure}
	\begin{subfigure}[h!]{0.3\textwidth}
		\centering
		\includegraphics[width=\textwidth]{app-ora_decision.png}
		\caption{apple-orange}
		\label{fig:apple-orange}
	\end{subfigure}
	\caption{Decision boundaries for each class pair.}
	\label{fig:boundaries}
\end{figure}

%\clearpage
\begin{table}[!htb]
	\centering
	\caption{Self-evaluation.}
	\begin{tabular}{||r|c||}
		\hline
		Technical correctness & 5 \\ \hline
		Quality of presentation & 5 \\ \hline
		Initiative & 0 \\ \hline
		\textbf{TOTAL} & \textbf{10} \\ \hline
	\end{tabular}
	\label{tab:self-eval}
\end{table}

\bibliographystyle{spp-bst}
\bibliography{biblio}

\end{document}