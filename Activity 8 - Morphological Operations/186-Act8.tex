\documentclass[12pt,a4paper]{article}
\usepackage{physics}
\usepackage{amssymb}
\usepackage{subcaption}
\usepackage{colortbl}
\newcommand{\activity}{Activity 8 -- Morphological Operations}
\input{spp.dat}

\begin{document}

\title{\TitleFont \activity}
\author[ ]{\textbf{Kenneth V. Domingo} \\
2015--03116 \\
App Physics 186, 1\textsuperscript{st} Semester, A.Y. 2019--20}
\affil[ ]{\corremail{kvdomingo@up.edu.ph} }

\maketitle
\thispagestyle{titlestyle}

\section*{Results and Discussion}
For this activity \cite{soriano}, we make use of several basic shapes, namely:

\begin{itemize}
	\item a $5 \times 5$ square
	\item a $4 \times 3$ triangle
	\item a hollow $10 \times 10$ box, 2 units thick
	\item a $5 \times 5$ plus sign, 1 unit thick
\end{itemize}

\noindent
to be manipulated by several structuring elements, namely:

\begin{itemize}
	\item $2 \times 2$ ones
	\item $2 \times 1$ ones
	\item $1 \times 2$ ones
	\item $3 \times 3$ cross
	\item $2 \times 2$ antidiagonal line
\end{itemize}

All the shapes were generated by Python's imaging library (PIL), and the operations were performed using OpenCV. For the shapes, their origin is located at the shape's geometric center, while for the structuring elements, their origin is located at the lower-left corner. The result for the dilation of the images is shown in Fig. \ref{fig:dilate}, while Fig. \ref{fig:erode} shows the result for erosion.

\begin{figure}[htb]
	\centering
	\includegraphics[width=\textwidth]{dilate.png}
	\caption{The result of dilation on the images. The first column shows the original images used, while the first row shows the structuring elements.}
	\label{fig:dilate}
\end{figure}

\begin{figure}[htb]
	\centering
	\includegraphics[width=\textwidth]{erode.png}
	\caption{The result of erosion on the images. The first column shows the original images used, while the first row shows the structuring elements.}
	\label{fig:erode}
\end{figure}

\clearpage
\begin{table}[!htb]
	\centering
	\caption{Self-evaluation.}
	\begin{tabular}{||r|c||}
		\hline
		Technical correctness & 5 \\ \hline
		Quality of presentation & 5 \\ \hline
		Initiative & 0 \\ \hline
		\textbf{TOTAL} & \textbf{10} \\ \hline
	\end{tabular}
	\label{tab:self-eval}
\end{table}

\bibliographystyle{spp-bst}
\bibliography{biblio}

\end{document}